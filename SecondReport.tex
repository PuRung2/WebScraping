% ------------------------------------------------------------------------------------
% ------------------------------------------------------------------------------------
% ------------------------------------------------------------------------------------
%
% 실험레포트용 양식. customized for EE
%

% document class
\documentclass[aps,10pt,a4paper]{article}

% packages
\usepackage[hangul]{kotex}
\usepackage{graphics}
\usepackage{fancyvrb,enumerate}
\usepackage{amsmath,amssymb,amscd,amsfonts}
\usepackage{geometry}

% 한글의 검색과 추출
\usepackage{ifpdf} % ifpdf 패키지를 불러준다.
\ifpdf
 \pdfmapfile{=unttf-pdftex-dhucs.map}
 \usepackage{dhucs-cmap}
 \usepackage[pdftex]{graphicx,color}
 \usepackage[unicode,pdftex]{hyperref}
\else
 \usepackage{graphicx,xcolor}
\fi

\usepackage{tikz}
\usetikzlibrary{shapes,arrows}

% 여백. geometry 패키지를 쓰면 된다.
\geometry {
	paper=a4paper,
	top=20mm,
	bottom=20mm,
	left=25mm,
	right=25mm
}

% 자주 사용되는 것들
\newcommand{\bg[1]}{\begin{#1}}
\newcommand{\dt}{\Delta t}
\newcommand{\tm}{\Delta t/2}
\newcommand{\vt}{v_t}

% chapter numbering format
% 이게 없으면 0.1.2 Reference 이런 식으로 나올 것임.
% 자세한 사항은 kotexguide.pdf를 읽어보도록 하시오.
\kscntformat{chapter}{}{.}
\kscntformat{section}{}{.}


% ------------------------------------------------------------------------------------
% ------------------------------------------------------------------------------------
% ------------------------------------------------------------------------------------
% ------------------------------------------------------------------------------------% ------------------------------------------------------------------------------------% ------------------------------------------------------------------------------------% ------------------------------------------------------------------------------------% ------------------------------------------------------------------------------------% ------------------------------------------------------------------------------------% ------------------------------------------------------------------------------------% ------------------------------------------------------------------------------------% ------------------------------------------------------------------------------------% ------------------------------------------------------------------------------------% ------------------------------------------------------------------------------------% ------------------------------------------------------------------------------------% ------------------------------------------------------------------------------------% ------------------------------------------------------------------------------------% ------------------------------------------------------------------------------------% ------------------------------------------------------------------------------------% ------------------------------------------------------------------------------------% ------------------------------------------------------------------------------------% ------------------------------------------------------------------------------------% ------------------------------------------------------------------------------------% ------------------------------------------------------------------------------------

\begin{document}

% 커버


\title{\huge{REPORT}}   % type title between braces



\author{ 오픈소스SW프로젝트 }         % type author(s) between braces
\date{2018.01.15 - 2018.01.18}    % type date between braces
\maketitle

\ \\ \\ \\ \\ 






\section{GitHub History}
\subsection{Master Branch Commit 전체적인 순}

\begin{enumerate}
\item Django 환경 조성
\item 파이썬 설치 및 셋팅
\item 스크래피 설치 및 셋팅
\item 웹스크래핑
\item Django 프로젝트 생성 
\item 기사 스크래핑
\item csv파일에 데이터 저장될 때 한글 깨짐 관련 오류 수정
\item json 파일로 기사 URL 저장하기
\item 부트 스트랩 적용
\end{enumerate}
\ \\ \\ \\ \\ \\ \\ \\ \\ \\ \\ \\ \\ \\ \\ \\ 

\section{Commit 주요  내역} 
\subsection{Django 환경 조성}

\begin{itemize}
\item WebScarpy 기능 구현이 완료 된다면 Django로 연동하여 웹에 띄워야하므로 미리 라이브러리 설치 및 테스트프로젝트 생성   
\end{itemize}

\subsection{파이썬 설치 및 셋팅}
\begin{itemize}
\item 파이썬 2.7.10 버전 설치할 때 주의사항과 설치 후 경로설정.    
\end{itemize}

\subsection{스크래피 설치 및 셋팅}
\begin{itemize}
\item Scrapy 프레임워크를 사용하기 위한 설치 및 셋팅인데 매우 중요하다. 하나라도 설치가 안되어 있는 경우 설치하라는 오류가 뜨며 실행되지 않기때문에 반드시 참고할 사항.\\다만 미리 언급은 했지만 2.7.x대의 파이썬버전 사용자가 참고할 사항이다.    
\end{itemize}

\subsection{웹스크래핑}
\begin{itemize}
\item 관련된 자료를 공부하며 부분적으로 구현했던 사항. 프로젝트의 목적에 맞는 부분으로는 아직 구현되기 전.    
\end{itemize}

\subsection{Django 프로젝트 생성}
\begin{itemize}
\item 연동할 프로젝트 생성.   
\end{itemize}

\subsection{기사 스크래핑}
\begin{itemize}
\item 다음(www.daum.net)의 특정 날짜의 여러 뉴스사의 기사제목과 url 데이터 수집    
\end{itemize}

\subsection{csv 파일에 데이터 저장될 때 한글 깨짐 관련 오류 수정}
\begin{itemize}
\item encoding 방식을 utf-8에서 euc-kr로 바꿨더니 깨져서 저장되었던 파일이 한글로 저장이 잘 되었다.
\end{itemize}

\subsection{json 파일로 기사 URL 저장하기}
\begin{itemize}
\item CSV파일과 JSON 파일 둘다 이용할 것은 아니지만 어떤걸 사용할지 몰라서 JSON파일로도 저장 시도. CSV파일과 마찬가지로 제목과 본문으로 가는 URL이 잘 저장되었다. 
\end{itemize}

\subsection{부트스트랩 적용}
\begin{itemize}
\item 웹 프론트엔드를 구현할 오픈소스 부트스트랩을 사용하기로 하고 사용할 템플릿을 골라 설정.    
\end{itemize}



























\end{document}